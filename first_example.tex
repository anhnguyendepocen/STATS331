\chapter{A First Example}
We will now look at a simple example that demonstrates all the features of
Bayesian statistics. The problem is quite simple, but we will be able to see
how we start with prior probabilities, and how exactly these get updated
into posterior probabilities after we get more information.


Two balls in a bag

Show JOINT view

Before we obtain the data, our uncertainties about which hypothesis is true,
and about which data we will observe, can be represented by a {\it joint}
probability distribution.

\begin{table}
\begin{center}
\begin{tabular}{|c|c|c|}
\hline
\hline
\end{tabular}
\end{center}
\end{table}


The Bayes' Box looks like this:
\begin{table}[h!]
\begin{center}
\begin{tabular}{|c|c|c|c|c|}
\hline
{\bf Possible Hypotheses} & {\tt prior} & {\tt likelihood} &
{\tt prior $\times$ likelihood} & {\tt posterior}\\
\hline
{\tt BB} & 0.5 & 1   & 0.5  & 0.667\\
{\tt BW} & 0.5 & 0.5 & 0.25 & 0.333\\
\hline
Totals: & 1 & & 0.75 & 1\\
\hline
\end{tabular}
\end{center}
\end{table}

