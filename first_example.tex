\chapter{A First Example}

Here we will study a simple example to see what is going on with Bayesian
statistics. Suppose that there are two balls in a bag. We know in advance
that at least one of them is black. But we're not sure whether they're both
black, or whether one is black and one is white.

Consider the two hypotheses:

{\bf BB}: both balls are black\\
{\bf BW}: one is black and one is white.



Show JOINT view

Before we obtain the data, our uncertainties about which hypothesis is true,
and about which data we will observe, can be represented by a {\it joint}
probability distribution.

\begin{table}
\begin{center}
\begin{tabular}{cccccc}
 & \vline & {\tt B} & {\tt W} & \vline & \\
\hline
{\tt BB} & \vline & & & \vline & $1/2$\\
{\tt BW} & \vline & & & \vline & $1/2$\\
\hline
\end{tabular}
\end{center}
\end{table}
