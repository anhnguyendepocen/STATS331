\chapter{R Background}
This chapter describes the R programming techniques that we will need in order
to do Bayesian statistics both within R and also by using JAGS.

\section{Vectors}
We will use vectors frequently throughout the course. A vector can be thought
of as a list of numbers. In R, you can create a vector using the function
{\tt c}, which stands for ``concatenate''.
\begin{verbatim}
my_vector = c(12.4, -6.2, 4.04)
\end{verbatim}
You can then examine the contents of the vector by typing its name.
\begin{verbatim}
> my_vector
[1] 12.40 -6.20  4.04
\end{verbatim}
There are various helpful things you can do with vectors. You can get the length
of a vector (the number of elements in it) like so:
\begin{verbatim}
> length(my_vector)
[1] 3
\end{verbatim}
You can do arithmetic
with vectors too, if they're the same length. For example:
\begin{verbatim}
> c(1,2,3)*c(2,5,3)
[1]  2 10 9
\end{verbatim}
If you wanted, you could assign the output to a new variable.

\section{Lists}
Lists are a bit like vectors, in that they can contain a lot of information in
a single variable. But instead of just being numbers, lists can contain all
sorts of things. For example, suppose I want a variable/object in an R program
to represent a person. A person can have a name and an age. Here is how to make
a list:
\begin{verbatim}
a_person = list(name="Nicole", age=21)
\end{verbatim}
If you needed to extract certain elements from a list, you can do it using the
\$ operator in R. For example suppose I wanted to extract the {\tt name} variable
from within {\tt a\_person}. I could do this:
\begin{verbatim}
> a_person$name
[1] "Nicole"
\end{verbatim}
and voila. When we use JAGS, a data set will be represented using a list.
So will our JAGS output.

If you have ever learned C, C++, or Matlab, a list in R is basically the same thing as
a ``struct'' in C/C++. If you have learned C++, Java, or Python, a list is
like a ``class'' but with just variables and no functions. In Python ``dictionaries''
are also a similar concept.

\section{Functions}
In this course you'll need to be able to read and understand simple R
functions, and perhaps write a few. A function is like a machine that takes
an input, does something, and returns an output. You have probably used many
built-in R functions already, like {\tt sum()}.

\begin{verbatim}
# Defining a function called my_function
my_function = function(x)
{
  # Do some stuff
  result = 3*x + 0.5
  return(result)
}
\end{verbatim}


\section{For Loops}
For loops are mostly used to repeat an action many times. Here is an example.
\begin{verbatim}
N = 100
for(i in 1:N)
{
  print(i)
}
\end{verbatim}

\section{Useful Probability Distributions}
Since R is a statistics program, it knows about a lot of probability
distributions already. So, if I wanted to use the probability density function
of a normal distribution, instead of having to code something like this:
\begin{verbatim}
f = exp(-0.5*((x - mu)/sigma)**2)/(sigma*sqrt(2*pi))
\end{verbatim}
I can just use the built-in function {\tt dnorm}.
\begin{verbatim}
f = dnorm(x, mean=mu, sd=sigma)
\end{verbatim}
Much easier! If, instead of wanting to evaluate the PDF, I wanted to generate
random samples from a normal distribution, I could use {\tt rnorm}.
\begin{verbatim}
# Generate 1000 samples
samples = rnorm(1000, mean=50., sd=10.)
\end{verbatim}

