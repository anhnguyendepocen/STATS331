\chapter{R Background}

This chapter describes the R programming techniques that we will need.

\section{Functions}

\section{Lists}

\section{For Loops}


\section{Useful Probability Distributions}
Since R is a statistics program, it knows about a lot of probability
distributions already. So, if I wanted to use the probability density function
of a normal distribution, instead of having to code something like this:
\begin{verbatim}
f = exp(-0.5*((x - mu)/sigma)**2)/(sigma*sqrt(2*pi))
\end{verbatim}
I can just use the built-in function {\tt dnorm}.
\begin{verbatim}
f = dnorm(x, mean=mu, sd=sigma)
\end{verbatim}
Much easier! If, instead of wanting to evaluate the PDF, I wanted to generate
random samples from a normal distribution, I could use {\tt rnorm}.
\begin{verbatim}
# Generate 1000 samples
samples = rnorm(1000, mean=50., sd=10.)
\end{verbatim}

