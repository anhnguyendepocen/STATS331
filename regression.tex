\chapter{Regression}

Linear regression is a very important topic in statistics. It is also a nice
example to demonstrate how Bayesian statistics works and how it is different
from classical or frequentist statistics.

The assumption is that the data are a list of measurements:
\begin{eqnarray}
D = \{y_1, y_2, ..., y_N\}.
\end{eqnarray}
The unknown parameters are:
\begin{eqnarray}
\boldsymbol{\theta} = \{\beta_0, \beta_1, \sigma\}.
\end{eqnarray}

The linear regression model says:
\begin{eqnarray}
y_i \sim \mathcal{N}\left(\beta_0 + \beta_1 x_i, \sigma^2\right).
\end{eqnarray}
This gives us the likelihood.




\chapter{Simple Linear Regression With Outliers}

