\chapter{ANOVA}
One-way ANOVA is a common procedure in statistics, and is related to the
simpler idea of a t-test. In this chapter we will look at the same questions
but will answer them using a Bayesian model instead of a classical statistical
test.

We will look at a simple example first, and then a more complex one.
The first example is based on one given in a 1976 article by physicist E. T. Jaynes,
called ``Confidence Intervals vs. Bayesian Intervals''. This is a fun read for
those who are interested in the battle between classical and Bayesian statistics
when the latter was making its comeback in the second half of the 20th century.
It's also where I got the crazy confidence interval example from.

Two manufacturers, $A$ and $B$, both make ``widgets'', and we are interested
in figuring out which manufacturer makes the best widgets (on average), as
measured by their lifetime. To determine this, we obtain 9 widgets from
manufacturer $A$ and 4 widgets from manufacturer $B$. The results for the
lifetimes are given below:
\begin{eqnarray}
y_A = \{ \}
\end{eqnarray}

Classically, this could be solved by using a $t$-test.

The underlying assumption of a $t$-test is that the data are normally
distributed around the mean values for each group.
