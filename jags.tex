\chapter{JAGS}
JAGS stands for ``Just Another Gibbs Sampler''. It is a program that allows the user
to implement Markov Chain Monte Carlo (MCMC) on fairly complicated problems
very quickly. Gibbs Sampling is a particular MCMC technique that is beyond the
scope of this course: however, it is not that different from the Metropolis-Hastings
method that we do study in this course. If you are faced with a statistics
problem that you would like to solve in a Bayesian way, JAGS makes it very
straightforward to implement a Bayesian model. Essentially, we just need to tell
JAGS the following things:
\begin{itemize}
\item The names of our parameters and the prior distributions
\item The likelihood
\end{itemize}
Then we simply load in the data and let it go!

In the early 1990s, WinBUGS was developed. Up until 2012, WinBUGS was the
program of choice for STATS 331. However, there are a number of disadvantages to
WinBUGS, so I decided to switch over to JAGS in 2013. The main advantages of
JAGS over WinBUGS are: i) it is open source software and works on all
operating systems, ii) it allows a more concise version of notation that can
save a lot of space and increase the readability of the code, and iii) it can
be called from R more easily than WinBUGS in my opinion. In addition to the
final point, a lot of time in previous iterations of 331 was spent teaching
different ways of using WinBUGS. In JAGS we only need to learn one way of using
it, which frees up time for us to concentrate on the stats!

\section{Basic JAGS Example}
It is instructive to go back to our very first parameter estimation problem:
the binomial likelihood with a uniform prior, our bus example. There we had a
single unknown parameter $\theta$, with a uniform prior between 0 and 1. To make
it even simpler, let's ignore the data, so that the posterior is just the prior!
To implement a uniform prior in JAGS, the code looks like this:
\begin{framed}
\begin{verbatim}
model
{
    # Uniform prior for theta between 0 and 1
    theta ~ dunif(0, 1)
}
\end{verbatim}
\end{framed}
That's it! Note that comments can be written with a \# symbol, just like in R.
This should be saved in a text file somewhere on your computer, preferably in
the same directory as the R code that will actually launch JAGS. Note that
the whole specification of the problem we are solving must be wrapped inside
a {\tt model {   }} block. Eventually, the likelihood will go in there as well.

Let's deconstruct our single line of JAGS code, {\tt theta \~{ } dunif(0, 1)}. The
first thing is simply the name of our parameter. We are free to name it as we
wish. The tilde sign implies statistician's notation is being used: we are
about to specify a probability distribution that applies to {\tt theta}. Finally,
the actual distribution is given: a uniform distribution between 0 and 1. Note
that the uniform distribution is {\tt dunif} and not {\tt uniform}, like how
there is a {\tt dunif} function in R. Therefore, all our line does is tell JAGS
that there is a variable called {\tt theta} and it has a uniform prior.

R code for running the MCMC is given below. Note that the {\tt make\_list()}
function is written by me and provided on Cecil.

\begin{framed}
\begin{verbatim}
# Load the rjags library
library('rjags')

# Load my make_list function
source('make_list.r')

# Create a JAGS model object
m = jags.model(file='model.txt', data=data)

# Do 10,000 steps for burn-in
update(m, 10000)

# Do 100,000 MCMC steps and save every 10th theta value
draw = jags.samples(m, 100000, thin=10, variable.names = c('theta'))

# Convert to a list
results = make_list(draw)
\end{verbatim}
\end{framed}

When this code is executed, the object {\tt results} is a list containing a
vector for each variable that was ``monitored'' by listing its name in
{\tt variable.names}. To plot a trace plot of the MCMC run, we can simply do
the following:
\begin{framed}
\begin{verbatim}
plot(results$theta, type='l')
\end{verbatim}
\end{framed}
You could look at the posterior (which is the same as the prior here) using a
histogram, and you can compute summaries using the methods discussed earlier.

