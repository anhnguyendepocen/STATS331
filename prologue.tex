\section{Prologue}
This course was originally developed by Dr Wayne Stewart (formerly of The
University of
Auckland) and was first offered in 2009. I joined the faculty in July 2012
and took over the course from him. Wayne is a passionate Bayesian and advocate
for the inclusion of Bayesian statistics in the undergraduate curriculum.
While this edition of the course differs from Wayne's in some ways
\footnote{The differences are mostly cosmetic. 90\% of the content is the same.}
, I hope I am able to do the topic justice in an accessible way.

In this course we will use the following software:
\begin{itemize}
\item R ({\tt http://www.r-project.org/}) \\
\item JAGS ({\tt http://mcmc-jags.sourceforge.net/}) \\
\item R-Studio ({\tt http://www.rstudio.org/})
\end{itemize}
In the labs, our preferred editor will be R-Studio, although any text editor
can be used in principle. These programs are all free and open source software.
That is, they are free to use, share and modify. They should work on
virtually any operating system including the three most popular:
Microsoft Windows, Mac OS X and GNU/Linux. In previous editions of the course,
WinBUGS was used instead of JAGS. However, WinBUGS has not been updated for
several years, and only works on Microsoft Windows. The differences between
the two are fairly minor.
