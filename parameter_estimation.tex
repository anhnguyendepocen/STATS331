\chapter{Parameter Estimation}
One of the most important ways of using Bayes' rule is in the topic
called {\it parameter estimation}. Parameter estimation is a fairly common
situation in statistics. In fact, it is possible to interpret almost any
problem in statistics as a parameter estimation problem!

Firstly, what is a parameter? Well, it is just a fancy term for a quantity or
a number that is unknown. For example, how many people are currently in New
Zealand? Well, a Google search suggests 4.405 million. But that does not mean
there are {\bf exactly} 4,405,000 people. It could be a bit more or a bit
less. Maybe it is 4,405,323, or maybe it is 4,403,886. We don't really know.
We could call the true number of people in New Zealand right now $\theta$.





Discrete
Bayes Box

Bus example

\begin{table}
\begin{center}
\begin{tabular}{|c|c|c|c|c|}
\hline
\tt{Possible Values} & \tt{Prior} & \tt{Likelihood} & \tt{h} & \tt{Posterior}\\
$\theta$ & $p(\theta)$ & $p(x|\theta)$ & $p(\theta)p(x|\theta)$ & $p(\theta|x)$\\
\hline
0 & 0.0909 & & &\\
0.1 & 0.0909 & & &\\
0.2 & 0.0909 & & &\\
0.3 & 0.0909 & & &\\
0.4 & 0.0909 & & &\\
0.5 & 0.0909 & & &\\
0.6 & 0.0909 & & &\\
0.7 & 0.0909 & & &\\
0.8 & 0.0909 & & &\\
0.9 & 0.0909 & & &\\
1 & 0.0909 & & &\\
\hline
Totals & 1 & & & 1\\
\hline
\end{tabular}
\end{center}
\end{table}

Continuous?


